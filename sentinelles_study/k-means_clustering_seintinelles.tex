% Options for packages loaded elsewhere
\PassOptionsToPackage{unicode}{hyperref}
\PassOptionsToPackage{hyphens}{url}
%
\documentclass[
]{article}
\usepackage{amsmath,amssymb}
\usepackage{iftex}
\ifPDFTeX
  \usepackage[T1]{fontenc}
  \usepackage[utf8]{inputenc}
  \usepackage{textcomp} % provide euro and other symbols
\else % if luatex or xetex
  \usepackage{unicode-math} % this also loads fontspec
  \defaultfontfeatures{Scale=MatchLowercase}
  \defaultfontfeatures[\rmfamily]{Ligatures=TeX,Scale=1}
\fi
\usepackage{lmodern}
\ifPDFTeX\else
  % xetex/luatex font selection
\fi
% Use upquote if available, for straight quotes in verbatim environments
\IfFileExists{upquote.sty}{\usepackage{upquote}}{}
\IfFileExists{microtype.sty}{% use microtype if available
  \usepackage[]{microtype}
  \UseMicrotypeSet[protrusion]{basicmath} % disable protrusion for tt fonts
}{}
\makeatletter
\@ifundefined{KOMAClassName}{% if non-KOMA class
  \IfFileExists{parskip.sty}{%
    \usepackage{parskip}
  }{% else
    \setlength{\parindent}{0pt}
    \setlength{\parskip}{6pt plus 2pt minus 1pt}}
}{% if KOMA class
  \KOMAoptions{parskip=half}}
\makeatother
\usepackage{xcolor}
\usepackage[margin=1in]{geometry}
\usepackage{color}
\usepackage{fancyvrb}
\newcommand{\VerbBar}{|}
\newcommand{\VERB}{\Verb[commandchars=\\\{\}]}
\DefineVerbatimEnvironment{Highlighting}{Verbatim}{commandchars=\\\{\}}
% Add ',fontsize=\small' for more characters per line
\usepackage{framed}
\definecolor{shadecolor}{RGB}{248,248,248}
\newenvironment{Shaded}{\begin{snugshade}}{\end{snugshade}}
\newcommand{\AlertTok}[1]{\textcolor[rgb]{0.94,0.16,0.16}{#1}}
\newcommand{\AnnotationTok}[1]{\textcolor[rgb]{0.56,0.35,0.01}{\textbf{\textit{#1}}}}
\newcommand{\AttributeTok}[1]{\textcolor[rgb]{0.13,0.29,0.53}{#1}}
\newcommand{\BaseNTok}[1]{\textcolor[rgb]{0.00,0.00,0.81}{#1}}
\newcommand{\BuiltInTok}[1]{#1}
\newcommand{\CharTok}[1]{\textcolor[rgb]{0.31,0.60,0.02}{#1}}
\newcommand{\CommentTok}[1]{\textcolor[rgb]{0.56,0.35,0.01}{\textit{#1}}}
\newcommand{\CommentVarTok}[1]{\textcolor[rgb]{0.56,0.35,0.01}{\textbf{\textit{#1}}}}
\newcommand{\ConstantTok}[1]{\textcolor[rgb]{0.56,0.35,0.01}{#1}}
\newcommand{\ControlFlowTok}[1]{\textcolor[rgb]{0.13,0.29,0.53}{\textbf{#1}}}
\newcommand{\DataTypeTok}[1]{\textcolor[rgb]{0.13,0.29,0.53}{#1}}
\newcommand{\DecValTok}[1]{\textcolor[rgb]{0.00,0.00,0.81}{#1}}
\newcommand{\DocumentationTok}[1]{\textcolor[rgb]{0.56,0.35,0.01}{\textbf{\textit{#1}}}}
\newcommand{\ErrorTok}[1]{\textcolor[rgb]{0.64,0.00,0.00}{\textbf{#1}}}
\newcommand{\ExtensionTok}[1]{#1}
\newcommand{\FloatTok}[1]{\textcolor[rgb]{0.00,0.00,0.81}{#1}}
\newcommand{\FunctionTok}[1]{\textcolor[rgb]{0.13,0.29,0.53}{\textbf{#1}}}
\newcommand{\ImportTok}[1]{#1}
\newcommand{\InformationTok}[1]{\textcolor[rgb]{0.56,0.35,0.01}{\textbf{\textit{#1}}}}
\newcommand{\KeywordTok}[1]{\textcolor[rgb]{0.13,0.29,0.53}{\textbf{#1}}}
\newcommand{\NormalTok}[1]{#1}
\newcommand{\OperatorTok}[1]{\textcolor[rgb]{0.81,0.36,0.00}{\textbf{#1}}}
\newcommand{\OtherTok}[1]{\textcolor[rgb]{0.56,0.35,0.01}{#1}}
\newcommand{\PreprocessorTok}[1]{\textcolor[rgb]{0.56,0.35,0.01}{\textit{#1}}}
\newcommand{\RegionMarkerTok}[1]{#1}
\newcommand{\SpecialCharTok}[1]{\textcolor[rgb]{0.81,0.36,0.00}{\textbf{#1}}}
\newcommand{\SpecialStringTok}[1]{\textcolor[rgb]{0.31,0.60,0.02}{#1}}
\newcommand{\StringTok}[1]{\textcolor[rgb]{0.31,0.60,0.02}{#1}}
\newcommand{\VariableTok}[1]{\textcolor[rgb]{0.00,0.00,0.00}{#1}}
\newcommand{\VerbatimStringTok}[1]{\textcolor[rgb]{0.31,0.60,0.02}{#1}}
\newcommand{\WarningTok}[1]{\textcolor[rgb]{0.56,0.35,0.01}{\textbf{\textit{#1}}}}
\usepackage{graphicx}
\makeatletter
\def\maxwidth{\ifdim\Gin@nat@width>\linewidth\linewidth\else\Gin@nat@width\fi}
\def\maxheight{\ifdim\Gin@nat@height>\textheight\textheight\else\Gin@nat@height\fi}
\makeatother
% Scale images if necessary, so that they will not overflow the page
% margins by default, and it is still possible to overwrite the defaults
% using explicit options in \includegraphics[width, height, ...]{}
\setkeys{Gin}{width=\maxwidth,height=\maxheight,keepaspectratio}
% Set default figure placement to htbp
\makeatletter
\def\fps@figure{htbp}
\makeatother
\setlength{\emergencystretch}{3em} % prevent overfull lines
\providecommand{\tightlist}{%
  \setlength{\itemsep}{0pt}\setlength{\parskip}{0pt}}
\setcounter{secnumdepth}{-\maxdimen} % remove section numbering
\ifLuaTeX
  \usepackage{selnolig}  % disable illegal ligatures
\fi
\IfFileExists{bookmark.sty}{\usepackage{bookmark}}{\usepackage{hyperref}}
\IfFileExists{xurl.sty}{\usepackage{xurl}}{} % add URL line breaks if available
\urlstyle{same}
\hypersetup{
  pdftitle={K-means clustering for seintinelles study},
  pdfauthor={Romaric Sallustre},
  hidelinks,
  pdfcreator={LaTeX via pandoc}}

\title{K-means clustering for seintinelles study}
\author{Romaric Sallustre}
\date{2023-08-31}

\begin{document}
\maketitle

Clustering is an important technique in Pattern Analysis to identify
distinct groups in data.K-means clustering is a type of unsupervised
learning, which is used when you have unlabeled data (i.e., data without
defined categories or groups). The goal of this algorithm is to find
groups in the data, with the number of groups represented by the
variable K. The algorithm works \emph{iteratively} to assign each data
point to one of the K groups based on the features that are provided.
Data points are clustered based on feature similarity. The results of
the K-means clustering algorithm are:

\begin{enumerate}
\def\labelenumi{\arabic{enumi}.}
\item
  The centroids of the K clusters, which can be used to label new data
\item
  Labels for the training data (each data point is assigned to a single
  cluster)
\end{enumerate}

Rather than defining groups before looking at the data, clustering
allows you to find and analyze the groups that have formed organically.
The ``Choosing K'' section below describes how the number of groups can
be determined.

Each \emph{centroid} of a cluster is a collection of feature values that
define the resulting groups. Examining the centroid feature weights can
be used to qualitatively interpret what kind of group each cluster
represents.

The goal of this file is to showcase K-means clustering for the likert
scale data in the seintinelles study.

\begin{Shaded}
\begin{Highlighting}[]
\NormalTok{df }\OtherTok{\textless{}{-}} \FunctionTok{read\_csv}\NormalTok{(}\StringTok{"likert\_data\_seintinelles.csv"}\NormalTok{) }
\end{Highlighting}
\end{Shaded}

\begin{verbatim}
## Rows: 61 Columns: 19
## -- Column specification --------------------------------------------------------
## Delimiter: ","
## dbl (19): use, organisation, acceptable, bse, recording, timeBSE, track, com...
## 
## i Use `spec()` to retrieve the full column specification for this data.
## i Specify the column types or set `show_col_types = FALSE` to quiet this message.
\end{verbatim}

\begin{Shaded}
\begin{Highlighting}[]
\NormalTok{df }\OtherTok{\textless{}{-}} \FunctionTok{na.omit}\NormalTok{(df)}
\NormalTok{df}
\end{Highlighting}
\end{Shaded}

\begin{verbatim}
## # A tibble: 61 x 19
##      use organisation acceptable   bse recording timeBSE track comfortRecrd
##    <dbl>        <dbl>      <dbl> <dbl>     <dbl>   <dbl> <dbl>        <dbl>
##  1     4            4          6     6         1       5     6            4
##  2     5            6          4     6         5       6     7            7
##  3     7            6          6     6         7       6     7            5
##  4     7            7          7     7         7       7     7            7
##  5     6            6          7     7         7       7     7            7
##  6     7            4          5     7         6       7     6            6
##  7     7            7          7     7         7       7     7            7
##  8     6            4          5     7         5       7     6            7
##  9     6            6          7     5         6       7     7            7
## 10     7            7          7     7         7       7     7            7
## # i 51 more rows
## # i 11 more variables: rightTimeBSE <dbl>, addInfo <dbl>, deleteData <dbl>,
## #   acceptableInfo <dbl>, intrusive <dbl>, useAgain <dbl>, expectation <dbl>,
## #   useful <dbl>, enjoy <dbl>, satisfaction <dbl>, recommend <dbl>
\end{verbatim}

\hypertarget{normalization-plot}{%
\subsection{Normalization Plot}\label{normalization-plot}}

Plotting of the correlation of the normalized values

\begin{Shaded}
\begin{Highlighting}[]
\CommentTok{\#install.packages("ggcorrplot")}
\FunctionTok{library}\NormalTok{(ggcorrplot)}
\NormalTok{df\_scale }\OtherTok{\textless{}{-}} \FunctionTok{scale}\NormalTok{(df)}
\FunctionTok{ggcorrplot}\NormalTok{(df\_scale)}
\end{Highlighting}
\end{Shaded}

\includegraphics{k-means_clustering_seintinelles_files/figure-latex/pressure-1.pdf}

\hypertarget{testing-for-clusters}{%
\subsection{Testing for Clusters}\label{testing-for-clusters}}

Trying out k-means with various clusters like 2, 3 and 4

\begin{Shaded}
\begin{Highlighting}[]
\NormalTok{k2 }\OtherTok{\textless{}{-}} \FunctionTok{kmeans}\NormalTok{(df, }\AttributeTok{centers =} \DecValTok{2}\NormalTok{, }\AttributeTok{nstart =} \DecValTok{20}\NormalTok{)}
\NormalTok{k3 }\OtherTok{\textless{}{-}} \FunctionTok{kmeans}\NormalTok{(df, }\AttributeTok{centers =} \DecValTok{3}\NormalTok{, }\AttributeTok{nstart =} \DecValTok{20}\NormalTok{)}
\NormalTok{k4 }\OtherTok{\textless{}{-}} \FunctionTok{kmeans}\NormalTok{(df, }\AttributeTok{centers =} \DecValTok{4}\NormalTok{, }\AttributeTok{nstart =} \DecValTok{20}\NormalTok{)}

\CommentTok{\# plots to compare}
\NormalTok{p2 }\OtherTok{\textless{}{-}} \FunctionTok{fviz\_cluster}\NormalTok{(k2, }\AttributeTok{geom =} \StringTok{"point"}\NormalTok{, }\AttributeTok{data =}\NormalTok{ df) }\SpecialCharTok{+} \FunctionTok{ggtitle}\NormalTok{(}\StringTok{"k = 2"}\NormalTok{)}
\NormalTok{p3 }\OtherTok{\textless{}{-}} \FunctionTok{fviz\_cluster}\NormalTok{(k3, }\AttributeTok{geom =} \StringTok{"point"}\NormalTok{,  }\AttributeTok{data =}\NormalTok{ df) }\SpecialCharTok{+} \FunctionTok{ggtitle}\NormalTok{(}\StringTok{"k = 3"}\NormalTok{)}
\NormalTok{p4 }\OtherTok{\textless{}{-}} \FunctionTok{fviz\_cluster}\NormalTok{(k4, }\AttributeTok{geom =} \StringTok{"point"}\NormalTok{,  }\AttributeTok{data =}\NormalTok{ df) }\SpecialCharTok{+} \FunctionTok{ggtitle}\NormalTok{(}\StringTok{"k = 4"}\NormalTok{)}


\FunctionTok{grid.arrange}\NormalTok{(p2, p3, p4, }\AttributeTok{nrow =} \DecValTok{2}\NormalTok{)}
\end{Highlighting}
\end{Shaded}

\includegraphics{k-means_clustering_seintinelles_files/figure-latex/unnamed-chunk-1-1.pdf}

\hypertarget{choosing-optimal-cluster}{%
\subsection{Choosing Optimal Cluster}\label{choosing-optimal-cluster}}

For choosing the right optimal cluster for the data. There are 3 usual
methods to measuring the quality of clustering

\begin{enumerate}
\def\labelenumi{\arabic{enumi}.}
\item
  Elbow
\item
  Silhouette
\item
  Gap-statistic
\end{enumerate}

\hypertarget{elbow-method}{%
\subsubsection{Elbow Method}\label{elbow-method}}

The \textbf{elbow} method plots the value of the cost function produced
by different values of~\emph{k}. As you know, if~\emph{k}~increases,
average distortion will decrease, each cluster will have fewer
constituent instances, and the instances will be closer to their
respective centroids. \textbf{Silhouette} method determines a point that
measures how close that point lies to its nearest neighbor points,
across all clusters. It provides information about clustering quality
which can be used to determine whether further refinement by clustering
should be performed on the current clustering.~

The disadvantage of the elbow and average~silhouette~methods is that
they measure a global clustering characteristic only. A more
sophisticated method is to use the gap statistic which~provides a
statistical procedure to formalize the elbow/silhouette heuristic in
order to estimate the optimal number of clusters.

\hypertarget{gap-statistics-method}{%
\subsubsection{Gap-Statistics Method}\label{gap-statistics-method}}

The gap statistic compares the total intracluster variation for
different values of~\emph{k}~with their expected values under the null
reference distribution of the data (i.e.~a distribution with no obvious
clustering). The reference dataset is generated using Monte Carlo
simulations of the sampling process. That is, for each variable in the
data set we compute its range {[}min(xi),max(xj){]}{]}~and generate
values for the n points uniformly from the interval min to max.

\begin{Shaded}
\begin{Highlighting}[]
\CommentTok{\# choosing optimal cluster}
\FunctionTok{set.seed}\NormalTok{(}\DecValTok{123}\NormalTok{)}
\CommentTok{\# three types of method for clustering 1{-}elbow, 2{-}silhouette, 3{-}gap\_staistic}
\FunctionTok{fviz\_nbclust}\NormalTok{(df, kmeans, }\AttributeTok{method =} \StringTok{"gap\_stat"}\NormalTok{) }\SpecialCharTok{+} \FunctionTok{labs}\NormalTok{(}\AttributeTok{subtitle =} \StringTok{"gap{-}statistic"}\NormalTok{) }
\end{Highlighting}
\end{Shaded}

\includegraphics{k-means_clustering_seintinelles_files/figure-latex/unnamed-chunk-2-1.pdf}

From the above plot, we can determine that the optimal cluster for the
Likert scale will be \textbf{6}.

\hypertarget{fitting-k-means}{%
\subsection{Fitting K-Means}\label{fitting-k-means}}

\begin{Shaded}
\begin{Highlighting}[]
\NormalTok{km.out }\OtherTok{\textless{}{-}} \FunctionTok{kmeans}\NormalTok{(df, }\DecValTok{6}\NormalTok{, }\AttributeTok{nstart =} \DecValTok{20}\NormalTok{)}
\NormalTok{km.out}
\end{Highlighting}
\end{Shaded}

\begin{verbatim}
## K-means clustering with 6 clusters of sizes 1, 19, 14, 3, 10, 14
## 
## Cluster means:
##        use organisation acceptable      bse recording timeBSE    track
## 1 4.000000     4.000000   6.000000 6.000000  1.000000     5.0 6.000000
## 2 6.894737     6.736842   7.000000 6.789474  6.789474     7.0 7.000000
## 3 6.285714     6.142857   6.714286 6.714286  6.714286     7.0 6.428571
## 4 6.333333     6.666667   7.000000 5.666667  6.666667     7.0 7.000000
## 5 6.400000     5.400000   5.100000 6.200000  5.900000     6.4 5.900000
## 6 6.000000     5.785714   6.357143 6.357143  5.928571     6.5 6.714286
##   comfortRecrd rightTimeBSE  addInfo deleteData acceptableInfo intrusive
## 1     4.000000     5.000000 7.000000   6.000000            5.0       1.0
## 2     6.947368     6.842105 6.842105   7.000000            7.0       1.0
## 3     7.000000     2.714286 6.214286   6.857143            7.0       1.0
## 4     7.000000     4.000000 5.000000   6.333333            5.0       1.0
## 5     6.000000     6.300000 6.200000   6.300000            6.5       2.4
## 6     6.642857     5.714286 6.428571   5.500000            6.5       1.0
##   useAgain expectation   useful    enjoy satisfaction recommend
## 1 1.000000    5.000000 5.000000 4.000000     4.000000  6.000000
## 2 6.736842    6.842105 6.947368 6.947368     6.842105  6.947368
## 3 6.285714    6.214286 6.785714 6.785714     6.785714  6.357143
## 4 5.666667    4.333333 4.666667 4.666667     4.333333  4.666667
## 5 4.300000    5.300000 6.000000 5.700000     5.500000  4.500000
## 6 6.500000    5.214286 6.500000 6.357143     6.214286  6.428571
## 
## Clustering vector:
##  [1] 1 6 5 2 3 5 2 3 4 2 4 2 3 3 3 2 5 3 4 6 5 5 5 6 2 3 6 2 5 6 2 6 3 2 6 2 5 6
## [39] 3 2 6 2 5 6 3 2 2 3 3 6 2 6 3 2 3 6 5 6 2 2 2
## 
## Within cluster sum of squares by cluster:
## [1]   0.00000  47.36842 137.00000  29.33333 195.50000 154.92857
##  (between_SS / total_SS =  51.0 %)
## 
## Available components:
## 
## [1] "cluster"      "centers"      "totss"        "withinss"     "tot.withinss"
## [6] "betweenss"    "size"         "iter"         "ifault"
\end{verbatim}

\begin{Shaded}
\begin{Highlighting}[]
\CommentTok{\# cluster vector}
\NormalTok{km.out}\SpecialCharTok{$}\NormalTok{cluster}
\end{Highlighting}
\end{Shaded}

\begin{verbatim}
##  [1] 1 6 5 2 3 5 2 3 4 2 4 2 3 3 3 2 5 3 4 6 5 5 5 6 2 3 6 2 5 6 2 6 3 2 6 2 5 6
## [39] 3 2 6 2 5 6 3 2 2 3 3 6 2 6 3 2 3 6 5 6 2 2 2
\end{verbatim}

\begin{Shaded}
\begin{Highlighting}[]
\CommentTok{\# class of the k{-}means}
\FunctionTok{str}\NormalTok{(km.out)}
\end{Highlighting}
\end{Shaded}

\begin{verbatim}
## List of 9
##  $ cluster     : int [1:61] 1 6 5 2 3 5 2 3 4 2 ...
##  $ centers     : num [1:6, 1:19] 4 6.89 6.29 6.33 6.4 ...
##   ..- attr(*, "dimnames")=List of 2
##   .. ..$ : chr [1:6] "1" "2" "3" "4" ...
##   .. ..$ : chr [1:19] "use" "organisation" "acceptable" "bse" ...
##  $ totss       : num 1152
##  $ withinss    : num [1:6] 0 47.4 137 29.3 195.5 ...
##  $ tot.withinss: num 564
##  $ betweenss   : num 587
##  $ size        : int [1:6] 1 19 14 3 10 14
##  $ iter        : int 4
##  $ ifault      : int 0
##  - attr(*, "class")= chr "kmeans"
\end{verbatim}

From the above results, we can see the the size of the 6 different
clusters, clustering vector and the list of various components like
centers, totss, etc,.

\texttt{\$\ cluster} indicates the cluster assignment for each of the 61
data points in your input data. For instance, the first data point is
assigned to cluster 3, the second data point is in cluster 5, the third
is in cluster 4, and so on.

\texttt{\$\ centers:\ num\ {[}1:6,\ 1:19{]}\ 6.89\ 6.33\ 4\ 6.4\ 6\ ...\ \ \ ..-\ attr(*,\ "dimnames")=List\ of\ 2\ \ \ ..\ ..\$\ :\ chr\ {[}1:6{]}\ "1"\ "2"\ "3"\ "4"\ ...\ \ \ ..\ ..\$\ :\ chr\ {[}1:19{]}\ "use"\ "organisation"\ "acceptable"\ "bse"\ ...}

This part shows the coordinates of the cluster centers in a
19-dimensional space (which corresponds to the 19 variables in your
original data). The matrix has 6 rows (one for each cluster) and 19
columns (one for each variable). The row names (``1'', ``2'', ``3'',
etc.) represent the cluster numbers, and the column names (``use'',
``organisation'', etc.) represent the variable names.

\texttt{\$\ withinss:\ num\ {[}1:6{]}\ 47.4\ 29.3\ 0\ 195.5\ 154.9\ ..}-
This part displays the sum of squared distances of data points within
each cluster to their respective cluster centers. For example, the sum
of squared distances for the first cluster is 47.4, for the second
cluster it's 29.3, and so on.

\texttt{\$\ tot.withinss:\ num\ 564}- The total within-cluster sum of
squares is 564, which is the sum of all the ``withinss'' values. This
represents the total variability of data points within all clusters.

\texttt{\$\ between} - The between-cluster sum of squares is 587, which
represents the sum of squared distances between the cluster centers and
the overall mean. It quantifies the variability between different
clusters.

\texttt{\$\ size:\ int\ {[}1:6{]}\ 19\ 3\ 1\ 10\ 14\ 14}- shows the
number of data points in each cluster. For instance, the first cluster
contains 19 data points, the second cluster contains 3 data points, and
so on.

\texttt{\$\ iter:\ int\ 4\ \ -}indicates that it took 4 iterations to
converge and reach the final clustering solution.

\texttt{\$\ ifault:\ int\ 0}- indicates that no faults were encountered
during the clustering process.

In summary, the output provides you with a detailed overview of the
results of the k-means clustering analysis you performed on your data.
It includes information about cluster assignments, cluster centers,
within-cluster variability, between-cluster variability, cluster sizes,
and convergence details.

\hypertarget{visualization}{%
\subsection{Visualization}\label{visualization}}

\begin{Shaded}
\begin{Highlighting}[]
\CommentTok{\# this provides a plot with cluster}
\FunctionTok{fviz\_cluster}\NormalTok{(km.out, df)}
\end{Highlighting}
\end{Shaded}

\includegraphics{k-means_clustering_seintinelles_files/figure-latex/unnamed-chunk-4-1.pdf}

\begin{Shaded}
\begin{Highlighting}[]
\FunctionTok{library}\NormalTok{(dplyr)}
\end{Highlighting}
\end{Shaded}

\begin{verbatim}
## 
## Attaching package: 'dplyr'
\end{verbatim}

\begin{verbatim}
## The following object is masked from 'package:gridExtra':
## 
##     combine
\end{verbatim}

\begin{verbatim}
## The following objects are masked from 'package:stats':
## 
##     filter, lag
\end{verbatim}

\begin{verbatim}
## The following objects are masked from 'package:base':
## 
##     intersect, setdiff, setequal, union
\end{verbatim}

\begin{Shaded}
\begin{Highlighting}[]
\NormalTok{df }\SpecialCharTok{\%\textgreater{}\%}
  \FunctionTok{mutate}\NormalTok{(}\AttributeTok{Cluster =}\NormalTok{ km.out}\SpecialCharTok{$}\NormalTok{cluster) }\SpecialCharTok{\%\textgreater{}\%}
  \FunctionTok{group\_by}\NormalTok{(Cluster) }\SpecialCharTok{\%\textgreater{}\%}
  \FunctionTok{summarise\_all}\NormalTok{(}\StringTok{"mean"}\NormalTok{)}
\end{Highlighting}
\end{Shaded}

\begin{verbatim}
## # A tibble: 6 x 20
##   Cluster   use organisation acceptable   bse recording timeBSE track
##     <int> <dbl>        <dbl>      <dbl> <dbl>     <dbl>   <dbl> <dbl>
## 1       1  4            4          6     6         1        5    6   
## 2       2  6.89         6.74       7     6.79      6.79     7    7   
## 3       3  6.29         6.14       6.71  6.71      6.71     7    6.43
## 4       4  6.33         6.67       7     5.67      6.67     7    7   
## 5       5  6.4          5.4        5.1   6.2       5.9      6.4  5.9 
## 6       6  6            5.79       6.36  6.36      5.93     6.5  6.71
## # i 12 more variables: comfortRecrd <dbl>, rightTimeBSE <dbl>, addInfo <dbl>,
## #   deleteData <dbl>, acceptableInfo <dbl>, intrusive <dbl>, useAgain <dbl>,
## #   expectation <dbl>, useful <dbl>, enjoy <dbl>, satisfaction <dbl>,
## #   recommend <dbl>
\end{verbatim}

The \textbf{\texttt{\%\textgreater{}\%}} pipe operator to chain together
a sequence of data manipulation operations. First, the
\textbf{\texttt{mutate()}} function is used to add a new column called
``Cluster'' to the Likert data, where each entry represents the cluster
assignment of the corresponding data point as determined by the k-means
algorithm stored in the \textbf{\texttt{km.out\$cluster}} vector. Next,
the \textbf{\texttt{group\_by()}} function is applied to group the data
by the ``Cluster'' column. Finally, the
\textbf{\texttt{summarise\_all("mean")}} function is used to calculate
the mean value of each variable within each cluster group.

\hypertarget{references}{%
\subsection{References}\label{references}}

\begin{itemize}
\item
  \url{https://delladata.fr/kmeans/}
\item
  \url{https://uc-r.github.io/kmeans_clustering}
\item
  \url{https://towardsdatascience.com/10-tips-for-choosing-the-optimal-number-of-clusters-277e93d72d92}
\end{itemize}

\end{document}
